\documentclass{article}
\usepackage{graphicx}
\usepackage{multirow}
\usepackage{geometry}
\usepackage{titlesec}

\usepackage{enumerate}

\geometry{top=1.3in, bottom=1.3in, left=1.2in, right=1.2in}
\setcounter{secnumdepth}{4}

\titleformat{\paragraph}
{\normalfont\normalsize\bfseries}{\theparagraph}{1em}{}
\titlespacing*{\paragraph}
{0pt}{3.25ex plus 1ex minus .2ex}{1.5ex plus .2ex}

\begin{document}

%page 1
\vspace*{3em}
\begin{center}
{\rule{11cm}{0.01cm} \\ \tiny~
\\ \large \sc{UM--SJTU ~Joint Institute \\ \tiny ~
\\ \large Introduction to Engineering \\ \tiny ~ 
\\ \large VG100 \\ \tiny ~}
\rule{11cm}{0.01cm}  
\vspace*{3.5em}
\\  \large\sc{Final Report on Project 2 \\ \tiny ~ \\ \huge Motion Tuner\\ ~ \\ \Large Group 3 Trinity}}
\end{center}

\normalsize
\vspace*{10em}
\begin{center}
\begin{tabular}{cl}
\large
\vspace{0.3em} Xie Shuxiang&516370910070\vspace{0.3em}\\ 
\large
Ma Kerui\vspace{0.3em} &516370910106 \vspace{0.3em}\\ 
\large
Wang Ren\vspace{0.3em} &516370910177 \vspace{0.3em}\\ 
\large
Zhu Boying\vspace{0.3em} &516370910165 \vspace{0.3em}\\ 
\large
Guo Chengzhang\vspace{0.3em} &516021910639 \vspace{0.3em}\\ 
 
\end{tabular}
\end{center}

\normalsize
\vspace*{1em}
\begin{center}
\large
$\hspace{1em}Instructor$\\
\vspace*{1em}
\begin{tabular}{lcl}
\large
Professor\hspace{0.5em}\vspace{0.3em} &\hspace{0.5em}Yanfeng Shen\hspace{0.5em}&\hspace{0.5em}Ph.D\\ 
Professor\hspace{0.5em} &\hspace{0.5em}Cynthia Vagnetti \hspace{0.5em}&\hspace{0.5em}Ph.D\\ 
\end{tabular}
\end{center}


%page1.5 mulu
\newpage
\tableofcontents


%%%%%%%%
%\newpage
\section{Executive Summary}


%%%%%%
%\newpage
\section{Acknowledgement}


%%%%%
%\newpage
\section{Introduction}

%%%%%%
%\newpage
\section{Problem}
\subsection{Statement of the problem}
\hspace*{2em}Many citizens find themselves under the pressure over moderate and too busy to pursue for happiness. They are in unhealthy mental conditions but lack for convenient ways to get happiness. People have tried to get happiness since the happiness lesson in Haward, aiming to teach people to be happy, has become one of the most popular courses. According to China Daily, many high school students find themselves under high pressure. Even though they want to do some exercise sometimes, the space is limited.\\
\hspace*{2em}We choose dance to help people relax because it is convenient and low cost. Compared to other healthy relaxing ways like jogging and walking, dancing can be indoors, not influenced by the weather or the air pollution. The cost for dance is quite low. Traditional dancing machines takes too much space and not suitable for work places and study rooms. What?s more, this kind of machines is unfriendly for dancing beginners.\\
�\hspace*{2em}As a result, traditional dancing machine cannot provide users with anytime and anywhere dance. \\
\subsection{Summary of the problem}
\hspace*{2em}In summary, people need ways to relax because they are busy and under pressure. Dance can help people to get happiness, but traditional dancing machine exists these problems:
\begin{enumerate}[\hspace*{3em}(1)  ]
\item Too much space
\item Unavailable in most places
\item Difficult for beginners
\end{enumerate}

%%%%%%%
\section{Need and Validation}
\subsection{Needs}
\hspace*{2em}Our product is to provide people with convenient ways to get happiness.
\hspace*{2em}To make our product convenient, the device should meet the needs of:
\begin{enumerate}[\hspace*{3em}(1)]
\item Portable: The device should be able to be carried anywhere, like an App or a bracelet
\item Low cost: The price should not be a burden for most people
\item Multi-function: There should be more than one kinds of motion detected and more than one types of music produced.
\end{enumerate}
\hspace*{2em}There are also some constrains:
\begin{enumerate}[\hspace*{3em}(1)]
\item Database: To make different functions, we need to build a large database to analyze, which may be too large for an iphone.
\item Budget: Smaller the chip, higher the price.
\end{enumerate}
\subsection{Validation}
\hspace*{2em}Our device can meet with the criteria:
\begin{enumerate}[\hspace*{3em}(1)]
\item For portable, our chip is only 1mm*1.5mm, which can be put into a bracelet.
\item For affordable, our device is only 130 RMB, which is only one thirds of the prices of dancing machines and one tenths of the prices of motion detectors in the market.  So, it will be affordable for most people. 
\item For multi-function, our chip will transmit 9 groups of acceleration per second, quickly enough for detecting motions and creating music. We also simplify 9 numbers to 1 to save space.
\end{enumerate}

%%%%%%
\section{Solution}

\subsection{Moblie Terminal Part}

\subsubsection{Flow Chart}

\hspace*{2em}First the motion of one person is detected by the sensor and become raw data.
   Then the raw data is filtered into acceleration data. 

   The filter mainly did two things.
   One is to abandon the useless data such as temperature and the other is to
   map the acceleration data into float point value between 0 to 5. 

   After that, the acceleration data will be processed through one of the
   following function, matcher and scaler. 

   After the processor, one sound track is created and finally the multiple
   sound tracks are mixed into the final audio. 

\subsubsection{Index Script}

The index script is the one that unity engine recognize and hand over its
control to. 
Thus, we use the index script to get access to the \textbf{physical} part of the
mobile phone, ask for \textbf{memory}, create \textbf{main loop}, and call for
\textbf{matcher} and \textbf{scaler} functions.
The flow chart of the index script is shown in Figure~\ref{indexScript}.

\begin{figure}[htbp]
\centering
\includegraphics[height=20cm]{figWR/a}
\caption{Flow Chart of the Index Script}
\label{indexScript}
\end{figure}

\subsubsection{Scaler algorithm}

   For the scaler part, also suppose we have the acceleration data.
   Then we separate them into 10 time intervals equally.
   And we find the max value in each interval.
   Then the ruler is created based on the max value and the min value of the
   acceleration data.
   The ruler create 7 blanks vertically for there is 7 tones in one music
   period, which are do re mi...
   Fill in the blank and finally a music score is created and it produces one
   sound track.

   The visual of the scaler process is shown in Figure~\ref{scalerStep}.

\begin{figure}[htbp]
\centering
\newcommand{\widthOfScalerStepFigure}{5cm}
\includegraphics[width=\widthOfScalerStepFigure]{figWR/scaler0}
\includegraphics[width=\widthOfScalerStepFigure]{figWR/scaler1}
\includegraphics[width=\widthOfScalerStepFigure]{figWR/scaler2}
\caption{Scaler Process}
\label{scalerStep}
\end{figure}

\subsubsection{Matcher algorithm}

   For the matcher part, suppose we have the acceleration data,
   then we compared it with three pre-configured answers,
   calculate the difference between answers and real data.
   The one that has the least sum of absolute value is the audio clip we select.
   Then the corresponding sound track is created.

   The visual of the matcher process is shown in Figure~\ref{matcherStep}.

\begin{figure}[htbp]
\centering
\newcommand{\widthOfMatcherFigure}{8cm}
\includegraphics[width=\widthOfMatcherFigure]{figWR/matcher1}
\includegraphics[width=\widthOfMatcherFigure]{figWR/matcher2}
\caption{Matcher Process}
\label{matcherStep}
\end{figure}

\subsubsection{Mixer}

   After we have multiple sound tracks through either of the previous process,
   we mix them together and create the final audio. Feel free to play it.

\subsection{PC Terminal}
\hspace*{2em}Besides the cellphone terminal part, we also designed a PC terminal part, because when users dance, cellphones in hands are not safe enough. Cellphones might be thrown out and hurt someone and then be broken. Moreover, the cellphones are too heavy to carry when doing some fierce action. So developing a safer and lighter bracelet is necessary. The second part of our product can exactly satisfy the needs. \\
\hspace*{2em}The second part consists of a bracelet and a PC terminal. The bracelet contains the sensor JY901, the bluetooth module and batteries. More specifically, the sensor JY901 can detect data, the bluetooth module can transfer data to the terminal and the batteries can power both JY901 and bluetooth. Overall, the bracelet mainly does the detecting job. On the other hand, the PC terminal mainly does the analyzing and generating jobs. The software on PC terminal developed by Unity3D can apply several different algorithm to analyze the data, so that the origin motion kind of the users can be defined. Then by comparing different conditions prepared previously, the software can mix and play all kinds audio source.  \\
\subsubsection{Hardware}
\paragraph{Attitude Sensor JY901}
\hspace*{2em}The sensor JY901 is a attitude sensor that can detect the acceleration, the angle of avertence and angular velocity. The sensor itself consists of three-axis gyroscope, three-axis acceleration sensors, three-axis digital compass and some other motion sensors. The sensor JY901 has the following characteristics:
\begin{itemize}
\item Flexible data outputting ports. (I2C, SPI, TTL are supported)
\item High speed data outputting rate. (Highest 500Hz)
\item Low power consumption. (17mA)
\item Short and stable initializing time. 
\item Support software development. 
\end{itemize}
\paragraph{Bluetooth Module}
\paragraph{Battery}
\paragraph{Fabrication}
\subsubsection{Software}
\paragraph{Data collection algorithm (same with the Mobile terminal part)}
\paragraph{Data transmission algorithm}
\paragraph{ Data analysis algorithm (same with the Mobile terminal part)}
\paragraph{Audio source generating algorithm (same with the Mobile terminal part)}
\subsubsection{Working Principles}

%%%%%%%%%%
%\newpage
\section{Objectives}
\subsection{Difficulties}
\hspace*{2em}Since we divided Part 4\&5 into two parts, the mobile terminal part and the PC terminal part, we will divide the Objective parts into these two parts as well. They share many general objectives. 

\subsubsection{General Difficulties }
\begin{itemize}
\item The determination of a matcher
\item A lack of standard audio source
\end{itemize}

\subsubsection{Mobile Terminal part}
\begin{itemize}
\item How to call the inner sensors in mobile phone
\item The calling of the speaker and the flashlight in mobile phone
\end{itemize}

\subsubsection{PC Terminal part}
\begin{itemize}
\item The difficulties to transform data from the sensor to the PC terminal
\end{itemize}



\subsection{Objectives of General Difficulties}
\subsection{Objectives of Mobile Terminal Part}
\subsection{Objectives of PC Terminal Part}

%%%%%%%
%\newpage
\section{Tasks}

%%%%%%%
%\newpage
\section{Schedule}

%%%%%
%\newpage
\section{Budget}


%%%%%%
%\newpage
\section{Key Personnel}

%%%%%
\newpage
\section{Conclusion}
\subsection{Current Achievement}
\hspace*{2em}Our product, motion tuner is able to detect users? motion and play corresponding music. The key part to achieve this goal is to analyze users? acceleration and create music with corresponding volume and tone. First, it transmits the information of uses? motion to terminals with the help of sensors. Then it analyzes the raw data from sensors and filters acceleration by serial protocol. Finally, it compares the acceleration with pre-set standards and creates matched sound tracks with acceleration. To create final audio, it is also able to mix different sound tracks to make music fluent. \\
\hspace*{2em}Our product is designed for everyone living under pressure in everyday life. It provides them with a new option for relaxation. Due to its portability and low cost, it is convenient and inexpensive for users to dance whenever and wherever. This advantage makes our product surpass traditional dancing machines. Also, its sensitivity creates a nice platform for dance lovers to practice by themselves and it always encourages users to explore new movements with unique music on their own. Therefore, it can even act as an instrument for dancers to give improvisation performances on stage.
\subsection{Future Development}
\hspace*{2em}In the future, we are going to focus on improving user-friendly design as well as bringing in more dancing modes. For user-friendly design, we will beautify the interface of our app to make it more elegant and attractive. Up to now we have two modes including power mode and story mode. In power mode, users are encouraged to create their own music by dancing freely. In story mode, users need to carry out a certain series of movements to the rhythm of played music. According to our plan, there will be more than three modes with different functions in our app to meet all the requirements of users.





\end{document}