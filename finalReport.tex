\documentclass{article}
\usepackage{graphicx}
\usepackage{multirow}
\usepackage{geometry}
\usepackage{titlesec}

\usepackage{enumerate}

\geometry{top=1.3in, bottom=1.3in, left=1.2in, right=1.2in}
\setcounter{secnumdepth}{4}

\titleformat{\paragraph}
{\normalfont\normalsize\bfseries}{\theparagraph}{1em}{}
\titlespacing*{\paragraph}
{0pt}{3.25ex plus 1ex minus .2ex}{1.5ex plus .2ex}

\begin{document}

\begin{center}
\includegraphics[width=10cm]{JILOGO}
\end{center}

%page 1
\begin{center}
    \vspace{1em}
{\rule{11cm}{0.01cm} \\ \tiny~
\\ \large \sc{UM--SJTU ~Joint Institute \\ \tiny ~
\\ \large Introduction to Engineering \\ \tiny ~ 
\\ \large VG100 \\ \tiny ~}
\rule{11cm}{0.01cm}  
\vspace*{1.5em}
\\  \large\sc{Final Report on Project 2 \\ \tiny ~ \\ \huge Motion Tuner\\ ~ \\ 
\includegraphics[width=6cm]{Logo}\\
\Large Group 3 Trinity}}\\

\end{center}


\normalsize
\vspace*{0.5em}
\begin{center}
    \large
\begin{tabular}{cl}
\large
\vspace{0.3em} Xie Shuxiang&516370910070\vspace{0.3em}\\ 
\large
Ma Kerui\vspace{0.3em} &516370910106 \vspace{0.3em}\\ 
\large
Wang Ren\vspace{0.3em} &516370910177 \vspace{0.3em}\\ 
\large
Zhu Boying\vspace{0.3em} &516370910165 \vspace{0.3em}\\ 
\large
Guo Chengzhang\vspace{0.3em} &516021910639 \vspace{0.3em}\\ 
 
\end{tabular}
\end{center}

\normalsize
\vspace*{1em}
\begin{center}
\large
$\hspace{1em}Instructor$\\
\vspace*{1em}
\begin{tabular}{lcl}
\large
Professor\hspace{0.5em}\vspace{0.3em} &\hspace{0.5em}Yanfeng Shen\hspace{0.5em}&\hspace{0.5em}Ph.D\\ 
Professor\hspace{0.5em} &\hspace{0.5em}Cynthia Vagnetti \hspace{0.5em}&\hspace{0.5em}Ph.D\\ 
\end{tabular}
\end{center}

\input{Summary}

\input{Acknowledge}


\newpage
\tableofcontents

\section{Introduction}
\par Our group is consisted of five students: Xie Shuxiang (group leader), Ma
Kerui, Wang Ren, Zhu Boying and Guo Chengzhang. Figure 1 shows a photo of us. 
\begin{figure}
    \centering
    \includegraphics[width=5in]{Pics/groupSym}
    \caption{Team members of TRINITY.}
    (From left to right: Xie Shuxiang, Wang Ren, Guo Chengzhang, Ma Kerui and
    Zhu Boying) 
\end{figure}

\par We are UM-SJTU JI freshmen who are taking VG100 course. And successfully
completing project 1, we focus our attention on designing a new product that can
help stressful people relax themselves. Since some of our members have great
interests and talents in dancing, we decide to explore for a convenient way for
daily dancing. Its function is to detect users' motion and create corresponding
music.

\section{Problems}

\subsection{Statement of the Problems}

Many citizens find themselves under the pressure over moderate and too busy to pursue for happiness. They are in unhealthy mental conditions but lack for convenient ways to get happiness. People have tried to get happiness since the happiness lesson in Haward, aiming to teach people to be happy, has become one of the most popular courses. According to China Daily, many high school students find themselves under high pressure, which is shown in the graph below. Even though they want to do some exercise sometimes, the space is limited.
\begin{figure}
    \centering
    \includegraphics[width=13cm]{Pics/Problems1}
    \caption{Proportion of Senior High School Students under High Pressure}
    \label{scalerStep}
\end{figure}



\input{need}

\section{Solution}

\subsection{Moblie Terminal Part}

\subsubsection{Flow Chart}

   First the motion of one person is detected by the sensor and become raw data.
   Then the raw data is filtered into acceleration data. 

   The filter mainly did two things.
   One is to abandon the useless data such as temperature and the other is to
   map the acceleration data into float point value between 0 to 5. 

   After that, the acceleration data will be processed through one of the
   following function, which are the Matcher and the Scaler. 

   After the the process of Matcher or Scaler function, one sound track is
   created and finally the multiple sound tracks are mixed into the final audio.

\subsubsection{Index Script}

The Unity Engine first recognize the index script and hand over its control to
it.  
Thus, we use the index script to get access to the 
\textbf{gyroscope}, \textbf{speaker} of the mobile phone, 
 ask for \textbf{memory}, create \textbf{main loop} (the loop in which the
 program will be in after initial setup), and call for 
\textbf{Matcher} and \textbf{Scaler} functions. 
The flow chart of the index script is shown in Figure~\ref{indexScript}.

\begin{figure}[H]
\centering
\includegraphics[height=20cm]{figWR/a}
\caption{Flow Chart of the Index Script}
\label{indexScript}
\end{figure}

\subsubsection{Scaler algorithm}

   For the scaler part, also suppose we have the acceleration data.
   Then we separate them into 10 time intervals equally.
   And we find the max value in each interval.
   Then the ruler is created based on the max value and the min value of the
   acceleration data.
   The ruler create 7 blanks vertically for there is 7 tones in one music
   period, which are ``do re mi fa so la si''.
   Fill in the blank and finally a music score is created and it produces one
   sound track.

   The visual of the scaler process is shown by step in Figure~\ref{scalerStep0}
   to Figure~\ref{scalerStep2}. 

\begin{figure}[H]
\centering
\newcommand{\widthOfScalerStepFigure}{5cm}
\includegraphics[width=\widthOfScalerStepFigure]{figWR/scaler0}
\caption{Scaler Process, original data}
\label{scalerStep0}
\end{figure}

\begin{figure}[H]
\centering
\newcommand{\widthOfScalerStepFigure}{5cm}
\includegraphics[width=\widthOfScalerStepFigure]{figWR/scaler1}
\caption{Scaler Process, split and create ruler}
\label{scalerStep1}
\end{figure}

\begin{figure}[H]
\centering
\newcommand{\widthOfScalerStepFigure}{5cm}
\includegraphics[width=\widthOfScalerStepFigure]{figWR/scaler2}
\caption{Scaler Process, final music score}
\label{scalerStep2}
\end{figure}


\subsubsection{Matcher algorithm}

   For the matcher part, suppose we have the acceleration data,
   then we compared it with three pre-configured answers,
   calculate the difference between answers and real data.
   The one that has the least sum of absolute value is the audio clip we select.
   Then the corresponding sound track is created.

   The visual of the matcher process is shown in Figure~\ref{matcherStep0} and Figure~\ref{matcherStep1}.

\begin{figure}[H]
\centering
\newcommand{\widthOfMatcherFigure}{8cm}
\includegraphics[width=\widthOfMatcherFigure]{figWR/matcher1}
\caption{Matcher Process, Wrong music}
\label{matcherStep0}
\end{figure}

\begin{figure}[H]
\centering
\newcommand{\widthOfMatcherFigure}{8cm}
\includegraphics[width=\widthOfMatcherFigure]{figWR/matcher2}
\caption{Matcher Process, Right music}
\label{matcherStep1}
\end{figure}

\subsubsection{Mixer}

   After we have multiple sound tracks through either of the previous process,
   we mix them together and create the final audio. Feel free to play it.

\subsection{PC Terminal}

Besides the cellphone terminal part, we also designed a PC terminal part,
because when users dance, cellphones in hands are not safe enough. Cellphones
might be thrown out and hurt someone and then be broken. Moreover, the
cellphones are too heavy to carry when doing some fierce action. So developing a
safer and lighter bracelet is necessary. The second part of our product can
exactly satisfy the needs.  

The second part consists of a bracelet and a PC terminal. The bracelet contains
the sensor JY901, the bluetooth module and batteries. More specifically, the
sensor JY901 can detect data, the bluetooth module can transfer data to the
terminal and the batteries can power both JY901 and bluetooth. Overall, the
bracelet mainly does the detecting job. On the other hand, the PC terminal
mainly does the analyzing and generating jobs. The software on PC terminal
developed by Unity3D can apply several different algorithm to analyze the data,
so that the origin motion kind of the users can be defined. Then by comparing
different conditions prepared previously, the software can mix and play all
kinds audio source.   


\subsubsection{Hardware}
\paragraph{Attitude Sensor JY901}

The sensor JY901 is a attitude sensor that can detect the
acceleration, the angle of avertence and angular velocity. The sensor itself
consists of three-axis gyroscope, three-axis acceleration sensors, three-axis
digital compass and some other motion sensors. The sensor JY901 has the
following characteristics: 

\begin{itemize}
\item Flexible data outputting ports. (I2C, SPI, TTL are supported)
\item High speed data outputting rate. (Highest 500Hz)
\item Low power consumption. (17mA)
\item Short and stable initializing time. 
\item Support software development. 
\end{itemize}
\paragraph{Bluetooth Module}
\paragraph{Battery}
\paragraph{Fabrication}
\subsubsection{Software}
\paragraph{Data collection algorithm (same with the Mobile terminal part)}
\paragraph{Data transmission algorithm}
\paragraph{Data analysis algorithm (same with the Mobile terminal part)}
\paragraph{Audio source generating algorithm (same with the Mobile terminal
  part)} 
\subsubsection{Working Principles}

\section{Objectives}
\subsection{Difficulties}
\hspace*{2em}Since we divided Part 4\&5 into two parts, the mobile terminal part
and the PC terminal part, we will divide the Objective parts into these two
parts as well. They share many general objectives. 

\subsubsection{General Difficulties}
\begin{itemize}
\item Have enough audio source to generate variable music
\item Produce a harmonic music
\end{itemize}

\subsubsection{Mobile Terminal part}
\begin{itemize}
\item How to call the inner sensors in mobile phone
\item The calling of the speaker and the flashlight in mobile phone 
\end{itemize}

\subsubsection{PC Terminal part}
\begin{itemize}
\item Transform data from the sensor to the PC terminal
\end{itemize}



\subsection{Objectives of General Difficulties}
\subsection{Objectives of Mobile Terminal Part}
\subsection{Objectives of PC Terminal Part}


\section{Tasks}

\subsection{General Tasks}

\subsubsection{Collecting and recording audio source}

In order to collect all kinds of audio source, we search the Internet to find
some short music with strong rhythm and can be looped. Since we have several
modes, different types of music may be useful. Moreover, we use the software
Garageband which consists of hundreds of audio source that we need. We record
each element we need in the software and then save them for future use. 

\subsubsection{Combine each audio source in a harmonic way}

Base on the theory of music such as the modal theory, we develop an algorithm
that can generate each audio source in a relatively harmonic way. Since it won’t
change the fact that if your motion is random and your music will also be
random, we can only try to reduce the effect of this problem. And further
development are still need in the future.  

\subsection{Mobile Terminal Tasks}

\subsubsection{Call the inner sensor in the iPhone}

The function of calling the sensor of iPhone has already been natively built.
We include a library and call a line of system function to get the sensor data,
which is in the form of 3-dimension vector.
The vector is preceessed later by separating its component in each direction.


\subsubsection{Call the speaker and flash light in the iPhone}

We use a group of Game Object which is designed to get access to cellphone's
hardware to call the speaker.
The object is very flexible, for we can change its volume, its source clip
easliy. 
We use Vufuria library (Widedly used in augmented reality field) to get easy
access to the flash light of iPhone.

\subsection{PC Terminal Tasks}

\subsubsection{Learn to transfer data from the sensor to the computer by
  bluetooth} 

First we need to learn the C\# language, which is used on the Unity platform.
Then search for information of serial port communication under the background of
C\#. 



\section{Schedule}
Our group's work on this project lasts for 40 days. The schedule shown in Figure 2 is divided into four main parts. The first three parts, detecting, analyzing and generating are corresponding to our software, which is the crucial part of our product. The final task is about further improvement, including testing and improving user-friendly design.
\begin{figure}
    \centering
    \includegraphics[width=13cm]{Gantt}
    \caption{Gantt Chart}
    \label{scalerStep}
\end{figure}


\section{Budget}
To detect user's motion, we need a sensitive sensor, a Bluetooth developing
board and two batteries to transmit the information of user's motion to
terminals. 
Table 1 shows the costs of our project.
Hyperlinks of these materials are listed in Appendix. 
The total budget is less than 150RMB.
The inexpensive price makes our product affordable for most people. 

\begin{center}
\begin{tabular}{|ccc|}
\hline Item&Number&Cost(RMB)\\
\hline JY901 sensor&x1&92.0\\
\hline Bluetooth developing board&x1&22.3\\
\hline Battery&x2&22.0\\
\hline Total&-&136.3\\
\hline
\end{tabular}
\vspace{0.5em}
~\\Table 1. Budget.\\
\end{center}


\section{Key Personnel}
\samepage
\begin{figure}[H]
    \centering
    \includegraphics[width=15cm]{Pics/Personnel}
    \caption{Key Personnel.}
\end{figure}


\section{Conclusion}
\subsection{Current Achievement}
\hspace*{2em}Our product, motion tuner is able to detect users? motion and play corresponding music. The key part to achieve this goal is to analyze users? acceleration and create music with corresponding volume and tone. First, it transmits the information of uses? motion to terminals with the help of sensors. Then it analyzes the raw data from sensors and filters acceleration by serial protocol. Finally, it compares the acceleration with pre-set standards and creates matched sound tracks with acceleration. To create final audio, it is also able to mix different sound tracks to make music fluent. \\
\hspace*{2em}Our product is designed for everyone living under pressure in everyday life. It provides them with a new option for relaxation. Due to its portability and low cost, it is convenient and inexpensive for users to dance whenever and wherever. This advantage makes our product surpass traditional dancing machines. Also, its sensitivity creates a nice platform for dance lovers to practice by themselves and it always encourages users to explore new movements with unique music on their own. Therefore, it can even act as an instrument for dancers to give improvisation performances on stage.
\subsection{Future Development}
\hspace*{2em}In the future, we are going to focus on improving user-friendly design as well as bringing in more dancing modes. For user-friendly design, we will beautify the interface of our app to make it more elegant and attractive. Up to now we have two modes including power mode and story mode. In power mode, users are encouraged to create their own music by dancing freely. In story mode, users need to carry out a certain series of movements to the rhythm of played music. According to our plan, there will be more than three modes with different functions in our app to meet all the requirements of users.





\end{document}